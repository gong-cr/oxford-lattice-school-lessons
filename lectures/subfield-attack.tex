% Created 2017-03-03 Fri 18:16
% Intended LaTeX compiler: pdflatex
\documentclass[presentation,smaller]{beamer}
\RequirePackage{etex}
\RequirePackage[l2tabu,orthodox]{nag}            %% Warn about obsolete commands and packages
\RequirePackage{amsmath,amsfonts,amssymb,amsthm} %% Math
\RequirePackage{ifxetex,ifluatex}                %% Detect XeTeX and LuaTeX
\RequirePackage{fixltx2e}                        %% provides \textsubscript
\RequirePackage{xspace}
\RequirePackage{graphicx}
\RequirePackage{comment}
\RequirePackage{url}
\RequirePackage{relsize}
\RequirePackage{booktabs}
\RequirePackage{tabularx}
\RequirePackage[normalem]{ulem}
\RequirePackage[all]{xy}
\RequirePackage{etoolbox}

%%%
%%% Code Listings
%%%

\RequirePackage{listings}
\lstdefinelanguage{Sage}[]{Python}{morekeywords={True,False,sage,cdef,cpdef,ctypedef,self},sensitive=true}

\lstset{frame=none,
  showtabs=False,
  showspaces=False,
  showstringspaces=False,
  commentstyle={\color{gray}},
  keywordstyle={\color{mLightBrown}\textbf},
  stringstyle ={\color{mDarkBrown}},
  frame=single,
  basicstyle=\tt\scriptsize\relax,
  backgroundcolor=\color{gray!190!black},
  inputencoding=utf8,
  literate={…}{{\ldots}}1,
  belowskip=0.0em,
}

\makeatletter
\patchcmd{\@verbatim}
  {\verbatim@font}
  {\verbatim@font\scriptsize}
  {}{}
\makeatother

%%%
%%% Tikz
%%%

\RequirePackage{tikz,pgfplots}

\usetikzlibrary{calc}
\usetikzlibrary{arrows}
\usetikzlibrary{automata}
\usetikzlibrary{positioning}
\usetikzlibrary{decorations.pathmorphing}
\usetikzlibrary{backgrounds}
\usetikzlibrary{fit,}
\usetikzlibrary{shapes.symbols}
\usetikzlibrary{chains}
\usetikzlibrary{shapes.geometric}
\usetikzlibrary{shapes.arrows}
\usetikzlibrary{graphs}

%% Cache

\ifdefined\tikzcaching  % chktex 1
  \usetikzlibrary{external}
  \tikzexternalize[prefix=build/]
  \tikzset{external/up to date check=diff}  %% MD5 fails from within emacs
\fi

%%%
%%% SVG (Inkscape)
%%%

\ifxetex % chktex 1
\newcommand{\executeiffilenewer}[3]{%
  {\immediate\write18{#3}} % hack
}
\else
\newcommand{\executeiffilenewer}[3]{%
  \ifnum\pdfstrcmp{\pdffilemoddate{#1}}%
    {\pdffilemoddate{#2}}>0%
    {\immediate\write18{#3}}
  \fi%
}
\fi

\newcommand{\includesvg}[2][1.0\textwidth]{%
 \executeiffilenewer{#1.svg}{#1.pdf}%
 {inkscape -z -D --file=#2.svg --export-pdf=#2.pdf --export-latex --export-area-page}%
 \def\svgwidth{#1} 
 \input{#2.pdf_tex}%
} 

%%%
%%% Metropolis Theme
%%%

\usetheme{metropolis}
\metroset{color/block=fill}
\metroset{numbering=none}
\metroset{outer/progressbar=foot}
\metroset{titleformat=smallcaps}

\setbeamercolor{description item}{fg=mLightBrown}
% \setbeamerfont{alerted text}{series=\bfseries}
\setbeamerfont{footnote}{size=\scriptsize}
\setbeamercolor{example text}{fg=mDarkBrown}

\renewcommand*{\UrlFont}{\ttfamily\smaller\relax}

%%%
%%% UTF-8
%%% 

\RequirePackage{unicodesymbols} % after metropolis which loads fontspec

%%%
%%% BibLaTeX
%%%

\RequirePackage[backend=bibtex,
            style=alphabetic,
            maxnames=4,
            citestyle=alphabetic]{biblatex}

\bibliography{local.bib,abbrev3.bib,crypto_crossref.bib,rfc.bib,jacm.bib}

\DeclareFieldFormat{title}{\alert{#1}}
\DeclareFieldFormat[book]{title}{\alert{#1}}
\DeclareFieldFormat[thesis]{title}{\alert{#1}}
\DeclareFieldFormat[inproceedings]{title}{\alert{#1}}
\DeclareFieldFormat[incollection]{title}{\alert{#1}}
\DeclareFieldFormat[article]{title}{\alert{#1}}
\DeclareFieldFormat[misc]{title}{\alert{#1}}

%%% 
%%% Microtype
%%%

\IfFileExists{upquote.sty}{\RequirePackage{upquote}}{}
\IfFileExists{microtype.sty}{\RequirePackage{microtype}}{}

\setlength{\parindent}{0pt}                   %%
\setlength{\parskip}{6pt plus 2pt minus 1pt}  %%
\setlength{\emergencystretch}{3em}            %% prevent overfull lines
\setcounter{secnumdepth}{0}                   %%

%%% Local Variables:
%%% mode: latex
%%% End:
\usepackage{graphicx}
\usepackage{grffile}
\usepackage{longtable}
\usepackage{wrapfig}
\usepackage{rotating}
\usepackage[normalem]{ulem}
\usepackage{amsmath}
\usepackage{textcomp}
\usepackage{amssymb}
\usepackage{capt-of}
\usepackage{hyperref}
\usepackage{microtype}
\usepackage{newunicodechar}
\usepackage[notions,operators,sets,keys,ff,adversary,primitives,complexity,asymptotics,lambda,landau,advantage]{cryptocode}
\usepackage{xspace}
\usepackage{units}
\usepackage{nicefrac}
\usepackage{gensymb}
\usepackage{amsthm}
\usepackage{amsmath}
\usepackage{amssymb}
\usepackage{xcolor}
\usepackage{listings}
\usepackage[color=yellow!40]{todonotes}
\newcommand{\cR}{\ensuremath{\mathcal{R}}\xspace}
\newcommand{\Z}{\ensuremath{\mathbb Z}\xspace}
\renewcommand{\C}{\ensuremath{\mathbb C}\xspace}
\newcommand{\R}{\ensuremath{\mathbb R}\xspace}
\newcommand{\K}{\ensuremath{\mathbb K}\xspace}
\renewcommand{\L}{\ensuremath{\mathbb L}\xspace}
\newcommand{\Q}{\ensuremath{\mathbb Q}\xspace}
\newcommand{\OK}{\ensuremath{\mathcal O_{\K}}\xspace}
\newcommand{\OL}{\ensuremath{\mathcal O_{\L}}\xspace}
\DeclareMathOperator{\Vol}{Vol}
\DeclareMathOperator{\Norm}{N}
\DeclareMathOperator{\Tr}{Tr}
\DeclareMathOperator{\Span}{Span}
\usetheme{default}
\author{Martin R. Albrecht}
\date{Oxford Lattice School}
\title{A subfield lattice attack on overstretched NTRU assumptions}
\hypersetup{
pdfauthor={Martin R. Albrecht},
pdftitle={A subfield lattice attack on overstretched NTRU assumptions},
pdfkeywords={},
pdfsubject={},
pdfcreator={Emacs 25.1.1 (Org mode 9.0.5)},
pdflang={English},
colorlinks,
citecolor=gray,
filecolor=gray,
linkcolor=gray,
urlcolor=gray
}
\begin{document}

\maketitle
\begin{frame}{Outline}
\tableofcontents
\end{frame}


\section{Introduction}
\label{sec:orgea42ddb}

\begin{frame}[label={sec:orgfb2f314}]{NTRUEncrypt}
\begin{description}
\item[{Key Generation}] \(\cR = \Z[X]/(X^n+1)\), modulus \(q\), width parameter \(σ\)
\begin{itemize}
\item Sample \(f \gets D_{\cR, σ}\) (invertible mod \(q\))
\item Sample \(g \gets D_{\cR, σ}\)
\item Publish \(h = {[g/f]}_q\)
\end{itemize}

\item[{Encrypt}] \(m ∈ \{0,1\}^n\)
\begin{itemize}
\item Sample \(s,e \gets D_{\cR,χ}, D_{\cR,χ}\)
\item Return \(2\, (h ⋅ s + e) + m\)
\end{itemize}

\item[{Decrypt}] \(c ∈ \cR_q\)
\begin{itemize}
\item \(m' = f ⋅ c = 2\, (g ⋅ s + f ⋅ e) + f ⋅ m\)
\item Return  \(m' \bmod 2 ≡ f ⋅ m \bmod 2\)
\end{itemize}
\end{description}

\note{Note
\begin{itemize}
\item point out χ
\end{itemize}}
\end{frame}

\begin{frame}[fragile,label={sec:orge7bdbde}]{The NTRU lattice \(\Lambda^q_h\)}
 \lstset{language=sage,label= ,caption= ,captionpos=b,numbers=none}
\begin{lstlisting}
sage: K.<zeta> = CyclotomicField(8)
sage: OK = K.ring_of_integers()
sage: h = -36*zeta^3 + 44*zeta^2 + 14*zeta + 28
sage: h
\end{lstlisting}

\(-36 \zeta_{8}^{3} + 44 \zeta_{8}^{2} + 14 \zeta_{8} + 28\)


\lstset{language=sage,label= ,caption= ,captionpos=b,numbers=none}
\begin{lstlisting}
sage: H = h.matrix(); q = 97
sage: block_matrix([[1, H],[0, q]])
\end{lstlisting}

\begin{small}
\(\left(\begin{array}{rrrr|rrrr}
1 &   &   &   & 28 & 14 & 44 & -36 \\
  & 1 &   &   & 36 & 28 & 14 & 44 \\
  &   & 1 &   & -44 & 36 & 28 & 14 \\
  &   &   & 1 & -14 & -44 & 36 & 28 \\
\hline
   &   &   &   & 97 &   &   &   \\
  &   &   &   &   & 97 &   &   \\
  &   &   &   &   &   & 97 &   \\
  &   &   &   &   &   &   & 97
\end{array}\right)\)
\end{small}
\end{frame}

\begin{frame}[label={sec:org85750ea}]{The NTRU lattice \(\Lambda^q_h\)}
\begin{itemize}
\item The lattice \(\Lambda^q_h\) defined by an NTRU instance for parameters \(\cR, q, σ\) has dimension \(2n\) and volume \(q^n\).
\item If \(h\) were uniformly random, the Gaussian heuristic predicts that the shortest vectors of \(\Lambda_h^q\) have norm \(≈ \sqrt{n q}\).
\item Whenever \[\Vert f \Vert ≈ \Vert g \Vert ≈ \sqrt{n}\,σ \ll \sqrt{n\,q},\] then \(\Lambda^q_h\) has \alert{unusually short vectors}.
\end{itemize}
\end{frame}

\begin{frame}[label={sec:org17dfeb5}]{NTRU}
\begin{definition}[NTRU Assumption]
It is hard to find a short vector in the \cR-module \[\Lambda_h^q = \{(x,y) \in \cR^2 \text{ s.t. } hx - y = 0 \bmod q\}\]
with \(\cR = \Z[X]/(P(X))\) and the promise that a short solution \((f,g)\) — the private key — exists.\footfullcite{HofPipSil96} \footfullcite{HofPipSil98}
\end{definition}
\end{frame}

\begin{frame}[label={sec:org6a11e82}]{NTRU Applications}
The NTRU assumption has been utilised for

\begin{itemize}
\item signatures schemes, \footfullcite{C:DDLL13}
\item fully homomorphic encryption,\footfullcite{STOC:LopTroVai12,IMA:BLLN13}
\item candidate constructions for multi-linear maps.\footfullcite{EC:GarGenHal13}
\end{itemize}
\end{frame}

\begin{frame}[label={sec:org7b4f2bb}]{Lattice Attacks}
\begin{itemize}
\item Recovering a short enough vector of some target norm \(τ\), potentially longer than \((f,g)\), is sufficient for an attack.\footfullcite{EC:CopSha97}

\item In particular, finding a vector \(o(q)\) would break many applications such as encryption.

\item This requires strong lattice reduction and NTRU remains asymptotically secure.\footfullcite{HofPipSil98} \footfullcite{EPRINT:HPSSWZ15a}
\end{itemize}
\end{frame}

\begin{frame}[label={sec:org0353fcd}]{Best Attacks}
\begin{description}
\item[{Practical}] combined lattice-reduction and meet-in-the-middle attack \footfullcite{HofSilWhy06} of Howgrave-Graham. \footfullcite{C:HowgraveGraham07} \footfullcite{EPRINT:Wunderer16}

\item[{Asymptotic}] BKW variant, with a heuristic complexity \(2^{\Theta(n/\log \log q)}\).\footfullcite{C:KirFou15}
\end{description}
\end{frame}

\section{Preliminaries}
\label{sec:org52288f7}

\begin{frame}[label={sec:org045e7ab}]{Cyclotomic Number Fields and Subfields}
\begin{itemize}
\item I’ll focus on Cyclotomic number rings of degree \(n = 2^k\) for ease of exposure, but everything can be made general.

\item Let \(\cR ≃ \Z[X]/(X^{n}+1)\) be the ring of integers of the Cylotomic number field \(\K = \Q(ζ_m)\) for some \(m=2^k\) and \(n = m/2\).

\item Let \(\L = \Q(ζ_{m'})\) with \(m' | m\) be a subfield of \(\K\).

\item The ring of integers of \(\L\) is \(\cR' ≃ \Z[X]/(X^{n'} + 1)\) with \(n' = m'/2\).

\item We write the canonical inclusion \(\cR' \subset \cR\) explicitly as \(L : \cR' \rightarrow \cR\).

\item The norm \(\Norm_{\K/\L}: \K \rightarrow \L\) is the multiplicative map defined by \[\Norm_{\K/\L} : f \mapsto  \prod_{\psi \in G'} \psi(f)\] where \(G'\) is the Galois subgroup corresponding to \(\L\).
\end{itemize}
\end{frame}

\begin{frame}[label={sec:org320437c}]{Geometry}
The ring \(\cR\) is viewed as a lattice by endowing it with the inner product \[\langle a , b\rangle = \sum_{i=0}^{n-1} a_i ⋅ b_i.\] 

\begin{itemize}
\item This defines a Euclidean norm denoted by \(\| \cdot \|\).

\item We will make use of the operator's norm \(|\cdot|\) defined by: \[|a| = \sup_{x \in \K^*} \|ax\|/\|x\| = \max |a_i|.\]

\item It holds that \(\| a⋅b \| ≤ |a| ⋅ \| b \|\) and \[|\Norm_{\K/\L}(a)| ≤ {|a|}^r ≤ {\|a\|}^r.\]
\end{itemize}
\end{frame}


\begin{frame}[label={sec:org6390a05}]{Lattice Reduction}
Lattice reduction algorithms produce vectors of length \[β^{Θ(n/β)} ⋅ λ_1(Λ)\] for a computational cost \[\poly \cdot 2^{\Theta(\beta)},\] with \(λ_1(Λ)\) the length of a shortest vector of \(Λ\).\footfullcite{AC:CheNgu11}
\end{frame}

\section{Subfield Lattice Attack}
\label{sec:org820daf2}
\begin{frame}[label={sec:org16235ab}]{Overview}
\begin{displaymath} 
  \xymatrix{
    \K = \Q(ζ_m)   & & & \\
    & \cR = \Z[ζ_m]  \ar@{-}[ul] & (h,{\color{mLightBrown} f,g}) \ar@{->}[dd]& (x,y) = {\color{mLightBrown}u⋅( f,g)}\\
    \L = \Q(ζ_{m'}) \ar@{-}[uu]   &  & &\\
    \Q \ar@{-}[u]   & \cR' = \Z[ζ_{m'}]  \ar@{-}[ul] \ar@{-}[uu] & (h',{\color{mLightBrown} f',g'})& (h' \rightarrow (x', y'))  \ar@{-}[l]  \ar@{->}[uu]\\
    &  \Z  \ar@{-}[u]  \ar@{-}[ul] & &
    }
\end{displaymath}
\end{frame}

\begin{frame}[label={sec:orgb33a85a}]{1. Norming Down}
Define \(f' = \Norm_{\K/\L}(f)\), \(g' = \Norm_{\K/\L}(g)\), and  \(h' = \Norm_{\K/\L}(h)\), then \((f',g')\) is a vector of \(\Lambda^q_{h'}\) and it may be an unusually short one.

\begin{table}[htbp]
\centering
\begin{tabular}{rrrrrrr}
\(n\) & \(\log q\) & \(r\) & \(\Vert f \Vert\) & \(\sqrt{2/3⋅n}\) & \(\Vert f' \Vert\) & \(\left(\sqrt{2/3⋅n} \right)^r\)\\
256 & 300 & 8 & 3.70893 & 3.70752 & 29.21967 & 29.66015\\
256 & 300 & 32 & 3.66546 & 3.70752 & 103.69970 & 118.64060\\
256 & 300 & 64 & 3.71731 & 3.70752 & 210.20853 & 237.28120\\
\end{tabular}
\caption{Observed norms, after relative norm operation. All norms are logs.}

\end{table}
\end{frame}

\begin{frame}[label={sec:org4c0c881}]{1. Norming Down}
We assume that the following lemma holds also for all reasonable distributions considered in cryptographic constructions:

Let \(f\) be sampled from spherical Gaussians of variance \(\sigma^2\). Then, \[\|f'\| \leq {\sqrt{n}}^{r-1} ⋅ {\|f\|}^r\]
\end{frame}

\begin{frame}[label={sec:orgc29170b}]{2. Lattice Reduction in the Subfield}
Run lattice reduction with block size \(β\) on lattice \(Λ_{h'}^q\), to obtain a vector \((x',y') ∈ Λ_{h'}^q\) with

\begin{tikzpicture}
  \begin{axis}[width=1.0\textwidth,
    height=0.8\textheight,
    ylabel={$\log \Vert (x',y') \Vert$},
    xlabel=$r$,
    xmin=1,xmax=12]
    \addplot[smooth,thick,mLightBrown,mark=] plot coordinates {
      ( 1.0, 187.53)  ( 1.2, 161.36)  ( 1.4, 143.45)  ( 1.6, 130.72)  ( 1.8, 121.43) 
      ( 2.0, 114.56)  ( 2.2, 109.44)  ( 2.4, 105.63)  ( 2.6, 102.84)  ( 2.8, 100.84) 
      ( 3.0,  99.48)  ( 3.2,  98.63)  ( 3.4,  98.21)  ( 3.6,  98.15)  ( 3.8,  98.38) 
      ( 4.0,  98.87)  ( 4.2,  99.57)  ( 4.4, 100.47)  ( 4.6, 101.52)  ( 4.8, 102.72) 
      ( 5.0, 104.05)  ( 5.2, 105.48)  ( 5.4, 107.02)  ( 5.6, 108.64)  ( 5.8, 110.35) 
      ( 6.0, 112.12)  ( 6.2, 113.96)  ( 6.4, 115.86)  ( 6.6, 117.81)  ( 6.8, 119.81) 
      ( 7.0, 121.85)  ( 7.2, 123.93)  ( 7.4, 126.05)  ( 7.6, 128.21)  ( 7.8, 130.40) 
      ( 8.0, 132.61)  ( 8.2, 134.85)  ( 8.4, 137.12)  ( 8.6, 139.41)  ( 8.8, 141.73) 
      ( 9.0, 144.06)  ( 9.2, 146.42)  ( 9.4, 148.79)  ( 9.6, 151.17)  ( 9.8, 153.58) 
      (10.0, 156.00)  (10.2, 158.43)  (10.4, 160.87)  (10.6, 163.33)  (10.8, 165.80) 
      (11.0, 168.28)  (11.2, 170.77)  (11.4, 173.27)  (11.6, 175.78)  (11.8, 178.30) 
};
  \end{axis}

  \draw (3.5,4.5) node {$\|(x',y')\| ≤ β^{Θ(2n'/β)} ⋅ λ_1(Λ_{h'}^q)$};
  \draw (4.3,3.7) node {$≤ β^{Θ(n/(β r))} ⋅ \|(f',g')\|$};
  \draw (4.25,2.9) node {$≤ β^{Θ(n/(β r))} ⋅ {(n\,σ)}^{o(r)}$};

\end{tikzpicture}
\end{frame}

\begin{frame}[label={sec:org479526c}]{The Right Kind of \((x',y')\)}
\((x',y')\) is a solution in the subfield, how could that be useful?

\pause

\begin{enumerate}
\item If \((x',y')\) is short enough, then it is an \(\cR'\)-multiple of \((f',g')\).
\item This will allow us to lift \((x',y')\) to a short vector in \(\Lambda^q_h\).
\end{enumerate}
\end{frame}

\begin{frame}[label={sec:org372573e}]{\((x',y') = v ⋅ (f',g')\)}
\begin{theorem}
Let \(f',g' \in \cR'\) be such that \(⟨ f'⟩\) and \(⟨ g'⟩\) are coprime ideals and that \(h'⋅ f' = g' \bmod q\) for some \(h' \in \cR'\). If \((x',y') \in Λ_{h'}^q\) has length verifying
\[\|(x',y') \| < \frac q {\|(f',g')\|},\]
then \((x',y') = v ⋅ (f',g')\) for some \(v ∈ \cR'\).
\end{theorem}
\end{frame}

\begin{frame}[label={sec:orgec191e6}]{3. Lifting the Short Vector}
To lift the solution from the sub-ring \(\cR'\) to \(\cR\) compute \((x,y)\) as
\begin{itemize}
\item \(x = L(x')\)  and
\item \(y = L(y') ⋅ h / L(h') \bmod q\),
\end{itemize}
where \(L\) is the canonical inclusion map.
\end{frame}

\begin{frame}[label={sec:org9d13319}]{Performance}
Can solve in time complexity \(\mathsf{poly}(n ) \cdot 2^{\Theta(\beta)}\) when

\begin{itemize}[<+->]
\item \alert{Direct lattice attack}: \(\beta / \log \beta = \Theta\left( n / \log q \right)\)
\item \alert{Subfield attack}: \(\beta / \log \beta = \Theta\left( n\, \log n / \log^2 q \right)\) whenever \(r = \Theta(\log q / \log n) > 1\)
\end{itemize}
\end{frame}


\begin{frame}[label={sec:org5ae2a28}]{Fin}
\begin{center}
\Huge \alert{Thank You}
\end{center}

\fullcite{EPRINT:AlbBaiDuc16}
\end{frame}
\end{document}