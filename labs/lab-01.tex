\documentclass[10pt,a4paper,nobib]{tufte-handout}
%\usepackage{fontspec}
\usepackage{graphicx}
\usepackage[utf8]{inputenc}
\usepackage[T1]{fontenc}
\usepackage{grffile}
\usepackage{longtable}
\usepackage{wrapfig}
\usepackage{rotating}
\usepackage[normalem]{ulem}
\usepackage{amsmath}
\usepackage{textcomp}
\usepackage{amssymb}
\usepackage{capt-of}
\usepackage{hyperref}
\usepackage{microtype}
\usepackage{newunicodechar}
\usepackage[notions,operators,sets,keys,ff,adversary,primitives,complexity,asymptotics,lambda,landau,advantage]{cryptocode}
\usepackage{xspace}
\usepackage{units}
\usepackage{nicefrac}
\usepackage{gensymb}
\usepackage{amsthm}
\usepackage{amsmath}
\usepackage{amssymb}
\usepackage{xcolor}
\usepackage{listings}
\usepackage[color=yellow!40]{todonotes}
%\newunicodechar{ }{~}
\newtheorem{lemma}{Lemma}
\newtheorem{theorem}{Theorem}
\newtheorem{definition}{Definition}
\newtheorem{remark}{Remark}
\newtheorem{corollary}{Corollary}

%% TODO NOTES

\newcommand{\malb}[2][inline]{\todo[#1]{\textbf{malb:} #2}\xspace}


% LaTeX header for org-mode lab assignments. 
%
% This file assumes
%
% - the standard org-mode prefix
% - #+LATEX_CLASS: handout
% - #+LATEX_CLASS_OPTIONS: [10pt,a4paper]
% - (setq org-latex-default-packages-alist
%      (remove* '("T1" "fontenc" t) org-latex-default-packages-alist :test 'equal))

\usepackage{ifxetex}

\ifxetex
  \usepackage{fontspec}
  \usepackage{unicodesymbols}
  \renewcommand{\textls}[2][5]{%
    \begingroup\addfontfeatures{LetterSpace=#1}#2\endgroup
  }
  \renewcommand{\allcapsspacing}[1]{\textls[15]{#1}}
  \renewcommand{\smallcapsspacing}[1]{\textls[10]{#1}}
  \renewcommand{\allcaps}[1]{\textls[15]{\MakeTextUppercase{#1}}}
  \renewcommand{\smallcaps}[1]{\smallcapsspacing{\scshape\MakeTextLowercase{#1}}}
  \renewcommand{\textsc}[1]{\smallcapsspacing{\textsmallcaps{#1}}}

\setmonofont[BoldFont={Cousine Bold},
             ItalicFont={Cousine Italic},
             BoldItalicFont={Cousine Bold Italic},
             Scale=MatchLowercase]{Cousine}

\fi

\RequirePackage[backend=bibtex,
            style=alphabetic,
            maxnames=4,
            citestyle=alphabetic]{biblatex}

\bibliography{local.bib,abbrev3.bib,crypto_crossref.bib}


\makeatletter
\patchcmd{\@verbatim}
  {\verbatim@font}
  {\verbatim@font\footnotesize}
  {}{}
\makeatother

\DeclareMathOperator{\Vol}{Vol}

\setlength{\parindent}{0em}
\setlength{\parskip}{1em}
\setlength{\RaggedRightParindent}{0em}

\usepackage{xcolor}

\lstdefinelanguage{Sage}[]{Python}{morekeywords={True,False,sage,cdef,cpdef,ctypedef,self},sensitive=true}

\lstset{showstringspaces=false,
        aboveskip=0.75em,
        basicstyle=\footnotesize\ttfamily,
        keywordstyle=\bfseries\color{gray!40!black},
        commentstyle=\itshape\color{gray!140!black},
        identifierstyle=\color{gray!40!black},
        stringstyle=\color{gray},
        backgroundcolor=\color{gray!190!black},
        frame=none,
        xleftmargin=\parindent}

\author{Martin R. Albrecht, Léo Ducas and Guillaume Bonnoron}
\date{22 March 2017}
\title{Lab 1\\\medskip
\large Constructions}
\hypersetup{
pdfauthor={Martin R. Albrecht, Guillaume Bonnoron and Léo Ducas},
pdftitle={Lab 1},
pdfkeywords={},
pdfsubject={},
pdfcreator={Emacs 25.1.1 (Org mode 9.0.5)},
pdflang={English},
colorlinks,
citecolor=gray,
filecolor=gray,
linkcolor=gray,
urlcolor=gray
}
\begin{document}

\maketitle
In this lab, we will implement elementary cryptographic primitives based on lattices, namely a public-key encryption scheme\footfullcite{STOC:Regev05} in a single bit version, multi-bit version and ring version\footfullcite{EC:LyuPeiReg10}. The key take-aways will be about random sampling in lattices, security evaluation and matrix manipulations. 

% \section{Introduction}
%
% As a warm up, we ask you to play with the included \textit{Learning With Errors} oracle from Sage that we will use below to contruct the schemes.
% \lstset{language=sage,label= ,caption= ,captionpos=b,numbers=none}
% \begin{lstlisting}
% sage: from sage.crypto.lwe import *
% \end{lstlisting}
%
% In this module you can find a generic LWE oracle, that allows you to choose all the parameters, together with oracles from the litterature: \texttt{Regev} or \texttt{LindnerPeikert} for example. We will first use the latter oracle
%
% \lstset{language=sage,label= ,caption= ,captionpos=b,numbers=none}
% \begin{lstlisting}
% sage: R = Regev(n=16)
% sage: R
% LWE(16, 257, Discrete Gaussian sampler over the Integers with sigma =
% 1.602003 and c = 257, 'uniform', None)
% \end{lstlisting}
%
% We can see that \texttt{R} is an LWE oracle in dimension 16 with modulus 257, an discrete Gaussian error distribution of parameter 1.602003 centered in 0, a uniform secret distribution and no limit on the number of samples. Now we can call it to generate LWE samples:
%
% \lstset{language=sage,label= ,caption= ,captionpos=b,numbers=none}
% \begin{lstlisting}
% sage: R2(), R2(), R2()
% (((211, 28, 215, 246, 237, 82, 72, 185, 130, 206, 34, 9, 237, 218, 245, 188),
%   236),
%  ((22, 171, 61, 96, 62, 167, 14, 241, 181, 105, 140, 199, 59, 214, 35, 174),
%   162),
%  ((244, 219, 129, 167, 13, 103, 20, 129, 174, 146, 4, 225, 214, 239, 36, 85),
%   154))
% \end{lstlisting}
%

\section{Introduction}
\label{sec:orgaad38e1}
As a warm up, we ask you to use the \textit{Discrete Gaussian Sampler} from Sage which follow the techniques described by Ducas et al\footfullcite{C:DDLL13}. First import the necessary class:

\lstset{language=sage,label= ,caption= ,captionpos=b,numbers=none}
\begin{lstlisting}
sage: from sage.stats.distributions.discrete_gaussian_integer \
import DiscreteGaussianDistributionIntegerSampler
\end{lstlisting}

As its explicit name suggests, you can now generate integers from this sampler, providing it with the standard deviation $\sigma$ of the distribution you wish to use.

\textbf{Note: } A Gaussian distribution with center $c$ and parameter $\sigma$ samples elements with a probability proportional to $\exp(-\pi\frac{(x-c)^2}{\sigma^2})$.

\lstset{language=sage,label= ,caption= ,captionpos=b,numbers=none}
\begin{lstlisting}
sage: D = DiscreteGaussianDistributionIntegerSampler(sigma=2.0, c=5)
sage: D
Discrete Gaussian sampler over the Integers with sigma = 2.0 and c = 5
\end{lstlisting}

This creates the sampler, that you can then call to produce your integers. Here they are centered around 5, and stddev 2.

\lstset{language=sage,label= ,caption= ,captionpos=b,numbers=none}
\begin{lstlisting}
sage: D(), D(), D(), D(), D(), D()
(6, 4, 2, 6, 3, 3)
\end{lstlisting}

\subsection{Public-key Encryption --- single bit}

The public-key encryption scheme we want to implement goes as follow. Given a dimension $n$, a modulus $q$, we work in $\mathbb{Z}^n_q = (\mathbb{Z}/q\mathbb{Z})^n$. We also need a discrete Gaussian distribution of stddev $\sigma$.

\begin{itemize}
  \item \underline{ppGen:} $\mathbf{A} \in \mathbb{Z}^{n\times n}_q$ a public uniformly random matrix.
  \item \underline{KeyGen:} The public key is $\mathbf{b}^t = \mathbf{s}^t\mathbf{A} + \mathbf{e}^t \mod q$, with $\mathbf{s}, \mathbf{e} \in \mathbb{Z}^n_q$ sampled from the Gaussian distribution. The secret key is $\mathbf{s}$.
  \item \underline{Enc:} To encrypt a bit $m$, consider $\mathbf{M} = \left[ \begin{smallmatrix}
    \mathbf{A} \\
    \mathbf{b}^t
  \end{smallmatrix}\right] \in \mathbb{Z}^{(n+1)\times n}_q$ and compute
  \begin{align*}
    \mathbf{c} &= \mathbf{M}\cdot\mathbf{x} + \left(
               \begin{smallmatrix}
                 \mathbf{0} \\
                 \mu \cdot \lfloor q/2 \rceil
               \end{smallmatrix} \right)\\
               &= \left(\begin{smallmatrix}
                 \mathbf{A}\cdot \mathbf{x} \\
                 \langle \mathbf{b}\cdot \mathbf{x}\rangle
               \end{smallmatrix} \right) +
               \left( 
               \begin{smallmatrix}
                 \mathbf{0} \\
                 \mu \cdot \lfloor q/2 \rceil
               \end{smallmatrix} \right) \in \mathbb{Z}^{n+1}_q
  \end{align*}
  where $\mathbf{x}$ is uniform in $\{0, 1\}^n$.
  \item \underline{Dec:}  To decrypt, compute: 
\begin{align*}
  (-\mathbf{s}, 1)^t \cdot \mathbf{c} &= (-\mathbf{s}, 1)^t \cdot \mathbf{M} \cdot \mathbf{x} + \mu \cdot \lfloor q/2 \rceil \\
                                      &= \langle \mathbf{e}^t \cdot \mathbf{x} \rangle + \mu \cdot \lfloor q/2 \rceil \\
                                      &\approx  \mu \cdot \lfloor q/2 \rceil
\end{align*}
If this is closer to 0 than to $\lfloor q/2 \rceil$ output 0, otherwise 1. It works if $\langle \mathbf{e} \cdot \mathbf{x} \rangle < q/4$ so $q$ and $\sigma$ should be chosen accordingly.
\end{itemize}

% In 2005, Regev introduced the \textit{Learning With Errors (LWE)} problem together with a public key encryption scheme. In this part our exercice will be to implement it (we recall below the details of the scheme). We have few parameters: a dimension $n$, a modulus $q$, an error distribution $\chi$ and a number of samples $m\approx (n+1) \log q$.
%
% \begin{itemize}
% \item \textbf{KeyGen:} The secret key $\mathbf{s}$ is a uniform random element in $\mathbb{Z}^n_q = (\mathbb{Z}/q\mathbb{Z})^n$ and the public key is generated by $m$ LWE samples $(\mathbf{a}_i, b_i = \langle \mathbf{s}, \mathbf{a}_i \rangle + e_i)$. A convenient way is to contruct the matrix $\mathbf{A} \in \mathbb{Z}^{m\times(n+1)}_q$ whose rows are the $\mathbf{a}_i$ and a column vector $\mathbf{b}$ whose coefficients are the $b_i$s. The public key is then $(\mathbf{A}, \mathbf{b})$.
% \item \textbf{Enc:} To encrypt a bit $\mu$, one chooses a uniform $\mathbf{x}\in \{0, 1\}^m$ and compute $$ \mathbf{c} = (\mathbf{x}\cdot \mathbf{A}, \langle \mathbf{x}\cdot \mathbf{b}\rangle) + (\mathbf{0}, \mu \cdot \lfloor q/2 \rceil ) \in \mathbb{Z}^{n+1}_q$$
% \item \textbf{Dec:} To decrypt, one computes: 
% \begin{align*}
%   (-\mathbf{s}, 1)^t \cdot \mathbf{c} &= (-\mathbf{s}, 1)^t \cdot \mathbf{A} \cdot \mathbf{x} + \mu \cdot \lfloor q/2 \rceil \\
%                                       &= \langle \mathbf{e} \cdot \mathbf{x} \rangle + \mu \cdot \lfloor q/2 \rceil \\
%                                       &\approx  \mu \cdot \lfloor q/2 \rceil
% \end{align*}
% If this is closer to 0 than to $\lfloor q/2 \rceil$ output 0, otherwise 1. It works if $\langle \mathbf{e} \cdot \mathbf{x} \rangle < q/4$ so $q$ and $\sigma$ should be chosen accordingly.
% \end{itemize}
%
\textbf{Ex 1:} Using the Gaussian sampler from above, implement the whole scheme i.e.~the public parameter and key generations, the encryption and the decryption.

\subsection{Public-key Encryption --- multi-bit}

To improve the efficiency of the encryption scheme, it is possible to encrypt multiple bits of plaintext in one ciphertext.\footfullcite{CCS:BCDMNN16} In the single bit setting, we had for 1 bit of plaintext: $n^2$ integers of public parameter, $n$ integers of public key and of secret key, and $n+1$ integers of ciphertext. With the multibit approach, we can encrypted $k^2$ bits for only a factor $k$ expansion in the size of the keys and ciphertexts.

\malb{This is older than Frodo, no?}

The generalization of the scheme is quite straightforward. The operations remain the same, the only differences is the shift from vectors of size $n$ to matrices of dimensions $n \times k$ for $\mathbf{s}, \mathbf{e}$ and $\mathbf{b}$ and $\mathbf{x}$. Thus $\mathbf{c}$ becomes a $(n+k) \times k$ matrix where the bottom $k\times k$ coefficients store the masked encryption of $k^2$ bits $\mu_{i,j}$.

\textbf{Ex 2:} Adapt your previous code to handle multiple plaintext bits.

\subsection{Public-key encryption --- ring setting}

As a last improvement, we will now shift our scheme from the generic lattice setting where $\mathbf{A}$ is uniformly random in $\mathbb{Z}^{n\times n}_q$ to an ideal lattice. For this, we will work in a polynomial ring, e.g.~$R = \mathbb{Z}_q[X]/(X^n+1)$ where $n$ is a power of 2 and $q$ is prime as before.

In this setting, the scheme is adapted as follow:
\begin{itemize}
  \item \underline{ppGen:} $\mathbf{a} \in R$ a public uniformly random polynomial of $R$.
  \item \underline{KeyGen:} The public key is $\mathbf{b} = \mathbf{s} \cdot \mathbf{a} + \mathbf{e} \in R$, with $\mathbf{s}, \mathbf{e} \in R$ sampled from the Gaussian distribution. The secret key is $\mathbf{s}$.
  \item \underline{Enc:} To encrypt a binary polynomial $\mathbf{m}$, pick random $\mathbf{r}, \mathbf{e'}, \mathbf{e''} \in R$ from the Gaussian distribution and compute
$$ (\mathbf{c}_0, \mathbf{c}_1) = (\mathbf{a}\cdot\mathbf{r} + \mathbf{e'}, \mathbf{b}\cdot \mathbf{r} + \mathbf{e''} + \mathbf{m} \cdot \lfloor q/2 \rceil) $$
  \item \underline{Dec:}  To decrypt, one computes: 
\begin{align*}
   \mathbf{c}_1 - s \cdot \mathbf{c}_0 &= \mathbf{m} \cdot \lfloor q/2 \rceil + \mathbf{b}\cdot \mathbf{r} - \mathbf{s}\cdot\mathbf{a}\cdot\mathbf{r} + \mathbf{e''} - \mathbf{s}\cdot\mathbf{e'}\\
                                       &= \mathbf{m} \cdot \lfloor q/2 \rceil + \mathbf{e}\cdot \mathbf{r} + \mathbf{e''} - \mathbf{s}\cdot\mathbf{e'} \\
                                       &\approx  \mathbf{m} \cdot \lfloor q/2 \rceil
\end{align*}
So for each coefficient we apply the same rule as before, if it is closer to 0 than to $\lfloor q/2 \rceil$ output 0, otherwise 1. 
\end{itemize}

\textbf{Ex 3:} For this adaptation, more code changes are needed. You can continue to use the integer sampler, but Sage comes with a polynomial sampler. The code snippet below shows you how to use polynomial ring in Sage and its Gaussiam Sampler over Polynomials.
\lstset{language=sage,label= ,caption= ,captionpos=b,numbers=none}
\begin{lstlisting}
from sage.stats.distributions.discrete_gaussian_polynomial \
import DiscreteGaussianDistributionPolynomialSampler
...

Zq = IntegerModRing(q)
Rq.<x> = Zq['x'].quotient_ring(x^n+1)
P = DiscreteGaussianDistributionPolynomialSampler(Rq, n, sigma)
P()
\end{lstlisting}


\section{Security evaluation}

Now that the scheme is working and fairly efficient, the question remains of its level of security. Here we have $n$ that determines both the modulus $q$ and the Gaussian parameter $\sigma$. So we will play with $n$, and later also with $q$ and $\sigma$, and explore the level of security that we obtain.

For this work, we will use the estimator\footfullcite{EPRINT:AlbPlaSco15} that models the performance of (nearly) all existing attacks against LWE\@. The project page presents a basic use of the estimator. The core components is the \texttt{estimate\_lwe()} function that computes estimated costs of several attacks (see the project page for the details on the attacks). This function takes as input the LWE parameters to assess : $n$ the dimension, $q$ the modulus and $\alpha = \sqrt{2\pi}\cdot \frac{\sigma}{q}$ which captures the error width with respect to $q$. 

\textbf{Ex 4:}
\begin{enumerate}
  \item Assess the security of your implementations above.
  \item What $n$ should you pick to have 80 bits of security? 128 bits? 256?
  \item Can you adapt the choice of $q$ and $\sigma$ in our code to improve the security while maintaining correctness?
\end{enumerate}

\textbf{Note:} Below a quick introduction on using the estimator.
\lstset{language=sage,label= ,caption= ,captionpos=b,numbers=none}
\begin{lstlisting}
sage: load("https://bitbucket.org/malb/lwe-estimator/raw/HEAD/estimator.py")
sage: set_verbose(1)
...
sage: _ = estimate_lwe(n, alpha, q, skip="arora-gb")
\end{lstlisting}

\begin{itemize}
	\item First, load the Sage Module. From a remote location like here, or from a local file.
\end{itemize}

\section{Example solutions}
\subsection{Public-key Encryption --- single bit}

\lstset{language=sage,label= ,caption= ,captionpos=b,numbers=none}
\begin{lstlisting}
from sage.stats.distributions.discrete_gaussian_integer import \
DiscreteGaussianDistributionIntegerSampler

class pke_singlebit():
  def __init__(self, dimension):
    self.n = dimension
    self.q = next_prime(self.n^2)
    self.sigma = sqrt(self.n/(2*pi.n()))
    self.D = DiscreteGaussianDistributionIntegerSampler(sigma=self.sigma)
    self.Zq = IntegerModRing(self.q)

  def pp_gen(self):
    self.A = random_matrix(self.Zq, self.n, self.n)

  def keygen(self):
    s = vector(self.Zq, [self.D() for _ in range(self.n)])
    e = vector(self.Zq, [self.D() for _ in range(self.n)])
    b = s * self.A + e
    return s, b

  def encrypt(self, m, pk):
    M = self.A.stack(pk)
    x = random_vector(self.n, 0, 2)
    c = M*x 
    c[self.n] = (c[self.n] + m*self.q//2) % self.q
    return c

  def decrypt(self, c, sk):
    d = list(-sk)
    d.append(1)
    m_dec = vector(d) * c
    return 1 if self.q//4 < m_dec and m_dec < (3*self.q)//4 else 0


dimension = 150
message = randint(0, 1)
scheme = pke_singlebit(dimension)
scheme.pp_gen()
sk, pk = scheme.keygen()
c = scheme.encrypt(message, pk)
m_dec = scheme.decrypt(c, sk)

print message
print m_dec
\end{lstlisting}

\subsection{Public-key Encryption - multi-bit}

\lstset{language=sage,label= ,caption= ,captionpos=b,numbers=none}
\begin{lstlisting}
from sage.stats.distributions.discrete_gaussian_integer import \
DiscreteGaussianDistributionIntegerSampler

class pke_multibit():
  def __init__(self, dimension, packing):
    self.n = dimension
    self.k = packing
    self.q = next_prime(self.n^2)
    self.sigma = sqrt(self.n/(2*pi.n()))
    self.D = DiscreteGaussianDistributionIntegerSampler(sigma=self.sigma)
    self.Zq = IntegerModRing(self.q)

  def pp_gen(self):
    self.A = random_matrix(self.Zq, self.n, self.n)

  def keygen(self):
    s = matrix(self.Zq, self.n, self.k, [self.D() for _ in range(self.n*self.k)])
    e = matrix(self.Zq, self.n, self.k, [self.D() for _ in range(self.n*self.k)])
    b = s.transpose() * self.A + e.transpose()
    return s, b

  def encrypt(self, m, pk):
    x = random_matrix(ZZ, self.n, self.k, x=2)
    m = zero_matrix(self.Zq, self.n, self.k).stack(m)
    M = self.A.stack(pk)
    c = (M*x + m * (self.q//2)) % self.q
    return c

  def decrypt(self, c, sk):
    d = -sk.transpose()
    d = d.augment(identity_matrix(self.k))
    m_dec = d * c
    f = lambda x: 1 if self.q//4 < x and x < (3*self.q)//4 else 0
    return m_dec.apply_map(f)


dimension = 150
packing = 4
message = random_matrix(ZZ, packing, x=2)

scheme = pke_multibit(dimension, packing)
scheme.pp_gen()
sk, pk = scheme.keygen()
c = scheme.encrypt(message, pk)
m_dec = scheme.decrypt(c, sk)

print message
print m_dec
\end{lstlisting}

\subsection{Public-key encryption - ring setting}


\lstset{language=sage,label= ,caption= ,captionpos=b,numbers=none}
\begin{lstlisting}
from sage.stats.distributions.discrete_gaussian_polynomial import \
 DiscreteGaussianDistributionPolynomialSampler

class pke_ring():
  def __init__(self, dimension):
    self.n = dimension
    self.q = next_prime(self.n^2)
    self.sigma = sqrt(self.n/(2*pi.n()))
    Zq = IntegerModRing(self.q)
    self.Rq = PolynomialRing(Zq, 'x').quotient_ring(x^dimension+1)
    self.P = DiscreteGaussianDistributionPolynomialSampler(self.Rq, self.n, self.sigma)

  def pp_gen(self):
    self.a = self.Rq.random_element()

  def keygen(self):
    s = self.P()
    e = self.P()
    b = s * self.a + e
    return s, b

  def encrypt(self, m, pk):
    r = self.P()
    c = (self.a*r + self.P(), pk*r + self.P() + self.Rq(m) * (self.q//2))
    return c

  def decrypt(self, c, sk):
    m_dec = c[1] - sk * c[0]
    f = lambda x: 1 if self.q//4 < x and x < (3*self.q)//4 else 0
    return map(f, m_dec.list())

dimension = 16
message = [randint(0, 1) for _ in range(dimension)]

scheme = pke_ring(dimension)
scheme.pp_gen()
sk, pk = scheme.keygen()
c = scheme.encrypt(message, pk)
m_dec = scheme.decrypt(c, sk)

print message
print m_dec
\end{lstlisting}

\section{Security evaluation}

\end{document}

%%% Local Variables:
%%% mode: latex
%%% TeX-engine: xetex
%%% TeX-master: t
%%% End:
